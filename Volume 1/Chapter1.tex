\documentclass[12pt, letterpaper]{article}

\usepackage{amsmath}
\usepackage{amssymb}

\usepackage[a4paper, total={6in, 8in}]{geometry}

\author{Adam Crowe}
\title{Shreve -- Stochastic Calculus for Finance, Vol. 1 \\ Chapter 1 Solutions}

\renewcommand{\arraystretch}{1.5}

\begin{document}

\maketitle

\vspace{5mm}
\noindent
\textbf{Problem 1.1} $\;$ We start with the assumption that $X_0 = 0$ and define:
\begin{align*}
    X_1 &\overset{\text{def}}= \Delta_0 S_1 + (1+r)(X_0 - \Delta_0 S_0) \\
    \implies \quad X_1 &= \left( S_1 - (1+r)S_0 \right) \Delta_0
\end{align*}
Depending on the outcome of the first toss, we have:
\begin{align*}
    X_1(H) = \left( u - (1+r) \right) S_0 \Delta_0, \quad
    X_1(T) = \left( d - (1+r) \right) S_0 \Delta_0
\end{align*}
Condition (1.1.2) states that $0 < d < 1 + r < u$, thus we have $u - (1+r) > 0$ and $d-(1+r) < 0$.
Therefore:
\begin{gather*}
    \text{sign}\left( X_1(H) \right) = \text{sign}\left( S_0 \Delta_0 \right) = - \text{sign}\left( X_1(T) \right)
\end{gather*}
If an outcome of the toss $\omega$ gives $X_1(\omega) > 0$, then $X_1(\bar \omega) < 0$, where $\bar \omega$ is the opposite outcome to $\omega$.
However $H$ and $T$ are assumed to have a positive probability, so if the probability that $X_1(\omega) > 0$ is positive, then the probability that $X_1(\bar \omega) < 0$ is also positive.
Thus Condition (1.1.2) precludes arbitrage.

\rightline{$\square$}

\vspace{5mm}
\noindent
\textbf{Problem 1.2} $\;$ We compute $X_1(H), X_1(T)$ using $S_1(H)=8, S_1(T)=2$:
\begin{gather*}
    X_1(H) = 8\Delta_0 + 3\Gamma_0 - \tfrac{5}{4}(4 \Delta_0 + 1.20 \Gamma_0) = 3 \Delta_0 + 1.5 \Gamma_0 \\
    X_1(T) = 2\Delta_0 + (0)\Gamma_0 - \tfrac{5}{4}(4 \Delta_0 + 1.20 \Gamma_0) = - 3 \Delta_0 - 1.5 \Gamma_0
\end{gather*}
Therefore $X_1(H) = - X_1(T)$.
By the same argument as in Problem 1.1, if there is a positive probability that $X_1 > 0$, then there is a positive probability that $X_1 < 0$ (assuming that both $H$ and $T$ have a positive probability of occuring).

\rightline{$\square$}

\vspace{5mm}
\noindent
\textbf{Problem 1.3} $\;$ We compute $V_0$ under (1.1.10) using $V_1(H) = S_1(H) = uS_0$ and $V_1(T) = S_1(T) = dS_0$:
\begin{align*}
    V_0 &= \frac{1}{1+r} \left[ \tilde p V_1(H) + \tilde q V_1(T) \right] \\
    &= \frac{1}{1+r} \left[ \tilde p (uS_0) + \tilde q (dS_0) \right] \\
    &= \frac{S_0}{1+r} \left[ \left( \frac{1+r - d}{u-d} \right) u + \left( \frac{u-1-r}{u-d} \right) d \right] \\
    &= \frac{S_0}{(1+r)(u-d)} \left( u +ur - ud + ud - d - dr \right) \\
    &= \frac{S_0}{(1+r)(u-d)} \left( u +ur - d - dr \right) \\
    &= S_0
\end{align*}
\rightline{$\square$}

\vspace{5mm}
\noindent
\textbf{Problem 1.4} $\;$ Let $\omega_1 \omega_2 \dots \omega_n$ be fixed. We compute:
\begin{align*}
    X_{n+1}(\omega_1 \omega_2 \dots \omega_n T) &= \Delta_n(\omega_1 \omega_2 \dots \omega_n) S_{n+1}(\omega_1 \omega_2 \dots \omega_n T) \\
    &+ (1+r) \left( X_n(\omega_1 \omega_2 \dots \omega_n)- \Delta_n(\omega_1 \omega_2 \dots \omega_n) S_n(\omega_1 \omega_2 \dots \omega_n) \right)
\end{align*}
We surpress the $\omega_1 \omega_2 \dots \omega_n$ and re-write as:
\begin{align*}
    X_{n+1}(T) &= \Delta_n S_{n+1}(T) + (1+r) \left( X_n- \Delta_n S_n \right) \\
    &= d \Delta_n S_n + (1+r) \left( X_n- \Delta_n S_n \right) \\
    &= \left(d-(1+r)\right)\Delta_n S_n + (1+r)X_n
\end{align*}
By the induction hypothesis, $X_n = V_n$.
We use the definition of $\Delta_n$ and $V_n$:
\begin{gather*}
    \Delta_n = \frac{V_{n+1}(H)-V_{n+1}(T)}{S_{n+1}(H)- S_{n+1}(T)} = \frac{V_{n+1}(H)-V_{n+1}(T)}{(u-d)S_n}
\end{gather*}
\begin{gather*}
    X_n = V_n = \frac{1}{1+r}\left[ \tilde p V_{n+1}(H) + \tilde q V_{n+1}(T)\right]
\end{gather*}
\begin{align*}
    \implies \quad X_{n+1}(T) &= \left(d-(1+r)\right) \left( \frac{V_{n+1}(H)-V_{n+1}(T)}{u-d} \right) + \left( \tilde p V_{n+1}(H) + \tilde q V_{n+1}(T) \right) \\
    &= -\tilde p \left(V_{n+1}(H)-V_{n+1}(T) \right) + \tilde p V_{n+1}(H) + \tilde q V_{n+1}(T) \\
    &= (\tilde p + \tilde q)V_{n+1}(T) = V_{n+1}(T)
\end{align*}
\rightline{$\square$}

\vspace{5mm}
\noindent
\textbf{Problem 1.5} $\;$ We first calculate $\Delta_1(H)$:
\begin{align*}
    \Delta_1(H) = \frac{V_2(HH)- V_2(HT)}{S_2(HH)-S_2(HT)} = \frac{3.20 - 2.40}{16-4} = \frac{0.8}{12} = \frac{1}{30}
\end{align*}
At time one, the agent has a portfolio valued at $V_1(H)=2.24$.
Using the wealth equation, we calculate:
\begin{gather*}
    X_{2}(HH) = (16)\left(\tfrac{1}{15}\right) + \tfrac{5}{4}\left[ (2.24) - \left(\tfrac{1}{15}\right)(8) \right] = \tfrac{16}{15} + \tfrac{5}{4}\cdot\tfrac{128}{75} = \tfrac{16}{5} = 3.20 = V_2(HH)\\
    X_{2}(HT) = (4)\left(\tfrac{1}{15}\right) + \tfrac{5}{4}\left[ (2.24) - \left(\tfrac{1}{15}\right)(8) \right] = \tfrac{4}{15} + \tfrac{5}{4}\cdot\tfrac{128}{75} = \tfrac{12}{5} = 2.40 = V_2(HT)
\end{gather*}
Next we calculate $\Delta_2(HT)$:
\begin{align*}
    \Delta_2(HT) = \frac{V_2(HTH)- V_2(HTT)}{S_2(HTH)-S_2(HTT)} = \frac{0-6}{8-2} = -1
\end{align*}
We have $X_2(HT)=V_2(HT)=2.40$:
\begin{gather*}
    X_{3}(HTH) = (8)(-1) + \tfrac{5}{4}\left[ (2.40) - (-1)(4) \right] = - 8 + \tfrac{5}{4}\cdot \tfrac{32}{5} = 0 = V_3(HTH) \\
    X_{3}(HTT) = (2)(-1) + \tfrac{5}{4}\left[ (2.40) - (-1)(4) \right] = - 2 + \tfrac{5}{4}\cdot \tfrac{32}{5} = 6 = V_3(HTT)
\end{gather*}

\rightline{$\square$}

\vspace{5mm}
\noindent
\textbf{Problem 1.6} $\;$ If at time zero, the bank purchases $\Delta_0$ shares, they must borrow $\Delta_0 S_0$ from the money market to finance this (if $\Delta_0<0$ this represents short position whose initial proceeds are invested in the money market).
At time one, the bank then has $X_1 = V_1 + \Delta_0 S_1 - (1+r)\Delta_0 S_0$:
\begin{gather*}
    X_1(H) = (3) + \Delta_0 (8) - \tfrac{5}{4}\Delta_0 (4) = 3 + 3\Delta_0 \\
    X_1(T) = (0) + \Delta_0 (2) - \tfrac{5}{4}\Delta_0 (4) = -3 \Delta_0
\end{gather*}
We see that if $\Delta_0 = -0.50$, then $X_1(H) = X_1(T) = 1.50 $, as desired.
Thus the bank should sell $0.50$ shares short at time zero.
This will net them proceeds of $(0.50)(4) = 2$, which they should invest into the money market.
The above calculations show that regardless of the outcome of the coin toss, the bank will have wealth $1.50$.

\vspace{5mm}
Alternatively, Theorem 1.2.2 tells us the amount of shares $\Delta_0'$ that the \textit{seller} of the option should buy at time zero in order to replicate the option:
\begin{gather*}
    \Delta_0 ' = \frac{V_1(H)-V_1(T)}{S_1(H)- S_1(T)} = \frac{3-0}{8-2} = 0.50
\end{gather*}
It follows, that the \textit{buyer} of the option (in this case the bank) should perform the opposite actions to the seller, in order to replicate the option.
Again, we see that the bank should sell $0.50$ shares short at time zero.

\rightline{$\square$}

\vspace{5mm}
\noindent
\textbf{Problem 1.7} $\;$ The payoff of a lookback option at time three is $V_3 = \max_{0 \leq n \leq 3} S_n - S_3$:
\begin{align*}
    V_3(HHH) &= 32 - 32 = 0 & V_3(THH) &= 8 - 8 = 0 \\
    V_3(HHT) &= 16 - 8 = 8 & V_3(THT) &= 4 - 2 = 2 \\
    V_3(HTH) &= 8 - 8 = 0 & V_3(TTH) &= 4 - 2 = 2 \\
    V_3(HTT) &= 8 - 2 = 6 & V_3(TTT) &= 4 - 0.50 = 3.50
\end{align*}
We know $\tilde p = \tilde q = \tfrac{1}{2}$, so then:
\begin{align*}
    V_2(HH) &= \tfrac{4}{5}\left[ \tfrac{1}{2}(0) + \tfrac{1}{2}(8) \right] = \tfrac{16}{5} = 3.20 & V_2(TH) &= \tfrac{4}{5}\left[ \tfrac{1}{2}(0) + \tfrac{1}{2}(2) \right] = \tfrac{4}{5} = 0.80 \\
    V_2(HT) &= \tfrac{4}{5}\left[ \tfrac{1}{2}(0) + \tfrac{1}{2}(6) \right] = \tfrac{12}{5} = 2.40  & V_2(TT) &= \tfrac{4}{5}\left[ \tfrac{1}{2}(2) + \tfrac{1}{2}(3.50) \right] = \tfrac{11}{5} = 2.20
\end{align*}
\begin{align*}
    V_1(H) &= \tfrac{4}{5}\left[ \tfrac{1}{2}(3.20) + \tfrac{1}{2}(2.40) \right] = \tfrac{56}{25} = 2.24 \\
    V_1(T) &= \tfrac{4}{5} \left[ \tfrac{1}{2}(0.80) + \tfrac{1}{2}(2.20) \right] = \tfrac{6}{5} = 1.20 \\
    V_0 &= \tfrac{4}{5} \left[ \tfrac{1}{2}(2.24) + \tfrac{1}{2}(1.20) \right] = \tfrac{172}{125} = 1.376
\end{align*}
We compute the values $\Delta_i$ for $i = 0, 1, 2$ according to Theorem 1.2.2:
\begin{align*}
    \Delta_0 = \frac{(2.24) - (1.20)}{(8) - (2)} = \frac{13}{75} = 0.173 
\end{align*}
\begin{align*}
    \Delta_1(H) = \frac{(3.20)-(2.40)}{(16)-(4)} = \frac{1}{30} = 0.033\quad \quad \Delta_1(T) = \frac{(0.80)-(2.20)}{(4)-(1)} = -\frac{7}{15} = -0.467
\end{align*}
\begin{align*}
    \Delta_2(HH) &= \frac{(0)-(8)}{(32) - (8)} = - \frac{1}{3} & \Delta_2(TH) &= \frac{(0)-(2)}{(8)-(2)} = - \frac{1}{4} \\
    \Delta_2(HT) &= \frac{(0)-(6)}{(8)-(2)} = -1 & \Delta_2(TT) &= \frac{(2)-(3.50)}{(2)-(0.50)} = -1
\end{align*}

As mentioned in Problem 1.6, the owner of the option should carry out the opposite actions determined by Theorem 1.2.2 in order to replicate the option.

\vspace{5mm}

The bank should at time zero, sell $0.173$ shares short, and invest the proceeds in the money market. We list the actions the bank should take depending on the result of the first two tosses:

\begin{center}
    \begin{tabular}{ c|c|c } 
        Outcomes & Time 1 & Time 2 \\
        \hline
        HH & Short 0.033 Shares & Buy 0.333 Shares  \\
        HT & Short 0.033 Shares  & Buy 1 Shares  \\
        TH & Buy 0.467 Shares & Buy 0.25 Shares  \\
        TT & Buy 0.467 Shares  & Buy 1 Shares 
    \end{tabular}
\end{center}

At any period, if the bank is shorting a stock, they should invest the proceeds in the money market, and if they are buying a stock, they should borrow from the money market to fund this.
Theorem 1.2.2 guarantees that if the bank follows the above strategy, at time 3, the portfolio will be worth $\left( \tfrac{5}{4} \right)^3 \cdot 1.376 = 2.6875$, regardless of the outcomes of the tosses.

\vspace{5mm}
\noindent
\textbf{Problem 1.8}

\vspace{5mm}
\noindent
(i)$\;$ We know that $V_n = \frac{1}{1+r}\left[\tilde p V_{n+1}(H)+\tilde q V_{n+1}(T)\right]$.
Futhermore $Y_{n+1} = Y_{n} + S_{n+1}$, hence:
\begin{align*}
    v_n(s,y) &= \tfrac{4}{5}\left[ \tfrac{1}{2}v_{n+1}(2s, y+2s) +\tfrac{1}{2}v_{n+1}\left(\tfrac{1}{2}s, y+\tfrac{1}{2}s\right) \right] \\
    &= \tfrac{2}{5} \left[ v_{n+1}(2s, y+2s) + v_{n+1}\left(\tfrac{1}{2}s, y+\tfrac{1}{2}s\right) \right]
\end{align*}

\rightline{$\square$}

\vspace{5mm}
\noindent
(ii)$\;$ Note that $v_3(s,y) = \left( \tfrac{1}{4}y-4 \right)^+$ only depends on $y$. The possible values for $y$ at time three are: 60, 36, 24, 18, 12, 9, 7.5. Also note that $v_3(s,y) = 0$ for $y \leq 16$:
\begin{gather*}
    v_3(s, 60) = \left(\tfrac{1}{4}(60)-4 \right)^+ = 11 \\
    v_3(s, 36) = \left(\tfrac{1}{4}(36)-4 \right)^+ = 5 \\
    v_3(s, 24) = \left(\tfrac{1}{4}(24)-4 \right)^+ = 2 \\
    v_3(s, 18) = \left(\tfrac{1}{4}(18)-4 \right)^+ = \tfrac{1}{2} \\
    v_3(s, 12) = v_3(s, 9) = v_3(s, 7.5) = 0
\end{gather*}
Then:
\begin{align*}
    v_2(16,28) &= \tfrac{2}{5}\left[v_{3}(32, 60) + v_{3}\left(8, 36\right)\right] = \tfrac{2}{5}((11)+(5)) = \tfrac{32}{5} \\
    v_2(4,16) &= \tfrac{2}{5}\left[v_{3}(8, 24) + v_{3}\left(2, 18\right)\right] = \tfrac{2}{5}\left((2)+\left(\tfrac{1}{2}\right)\right) = 1 \\
    v_2(4,10) &= \tfrac{2}{5}\left[v_{3}(8, 18) + v_{3}\left(2, 12\right)\right] = \tfrac{2}{5}\left(\left(\tfrac{1}{2}\right) + (0)\right) = \tfrac{1}{5} \\
    v_2(1,7) &= \tfrac{2}{5}\left[v_{3}(2, 9) + v_{3}\left(0.5, 7.5\right)\right] = \tfrac{2}{5}\left((0)+(0)\right) = 0
\end{align*}
\begin{align*}
    v_1(8, 12) &= \tfrac{2}{5}\left[v_{2}(16, 28) + v_{2}\left(4, 16\right)\right] = \tfrac{2}{5}\left(\left(\tfrac{32}{5}\right) + (1)\right) = \tfrac{74}{25} \\
    v_1(2, 6) &= \tfrac{2}{5}\left[v_{2}(4, 10) + v_{2}\left(1, 7\right)\right] = \tfrac{2}{5}\left(\left(\tfrac{1}{5}\right) + (0)\right) = \tfrac{2}{25} \\
\end{align*}
\begin{gather*}
    v_0(4,4) = \tfrac{2}{5}\left[ v_1(8,12) + v_1(2,6) \right] = \tfrac{2}{5}\left[ \left(\tfrac{74}{25}\right) + \left(\tfrac{2}{25}\right) \right] = \tfrac{152}{125} = 1.216
\end{gather*}

\rightline{$\square$}

\vspace{5mm}
\noindent
(iii)$\;$
\begin{align*}
    \delta_n(s,y) &= \frac{v_{n+1}(2s, y+2s) - v_{n+1}\left(\tfrac{1}{2}s, y+\tfrac{1}{2}s\right)}{(2s) - \left(\tfrac{1}{2}s\right)} \\ &= \frac{2}{3s}\left( v_{n+1}(2s, y+2s) - v_{n+1}\left(\tfrac{1}{2}s, y+\tfrac{1}{2}s\right) \right)
\end{align*}

\rightline{$\square$}

\vspace{5mm}
\noindent
\textbf{Problem 1.9} 

\vspace{5mm}
\noindent
(i)$\;$ 
We surpress $\omega_1 \omega_2 \dots \omega_n$ from notation for brevity.
For $0\leq n < N$, the wealth equation reads $X_{n+1} = \Delta_n S_{n+1} + (1+r_n)(X_n - \Delta_n S_n)$:
\begin{align*}
    X_{n+1}(H) &= u_n \Delta_n S_n + (1+r_n)(X_n - \Delta_n S_n) \tag{1}\\
    X_{n+1}(T) &= d_n \Delta_n S_n + (1+r_n)(X_n - \Delta_n S_n) \tag{2}
\end{align*}
We then define $\tilde p_n = \frac{1+r_n-d_n}{u_n-d_n}$ and $\tilde q_n = \frac{u_n-1-r_n}{u_n-d_n}$, which give $\tilde p_n + \tilde q_n = 1$. We multiply (1) by $\tilde p_n$ and (2) by $\tilde q_n$, then add:
\begin{gather*}
    \tilde p_n X_{n+1}(H)+ \tilde q_n X_{n+1}(T) = (\tilde p_n u_n + \tilde q_n d_n) \Delta_n S_n + (1+r_n)(X_n - \Delta_n S_n)
\end{gather*}
Notice that:
\begin{gather*}
    \tilde p_n u_n + \tilde q_n d_n = \frac{1}{u_n-d_n} \left( (1+ r_n )u_n - u_n d_n + u_n d_n - (1 + r_n)d_n \right) = 1 + r_n
\end{gather*}
Hence:
\begin{gather*}
    \tilde p_n X_{n+1}(H)+ \tilde q_n X_{n+1}(T) = (1+r_n)X_n \\
    \implies \quad X_n = \frac{1}{1+r_n} \left[ \tilde p_n X_{n+1}(H)+ \tilde q_n X_{n+1}(T)\right]
\end{gather*}
Note that both the LHS and RHS are functions of $\omega_1 \omega_2 \dots \omega_n$, so our result is consistent.

\vspace{5mm}

Thus if an agent buys regardless of the number shares the buy or short at time $n$, their wealth at time $n$ can be determined by the above formula.
If $V_n$, the price of a derivative security at time $n$ (prior to the expiration at time $N$), was greater than $X_N$,
then the agent could sell the derivative security, replicate the portfolio, and have a surplus of cash to invest in the money market.
In this case, an arbitrage exists.
Similarly, if $V_n < X_n$, then one could buy the derivative security for less than their portfolio, and again have a surplus of cash.
Thus it follows that $V_n = X_n$ for all $0 \leq n \leq N$.

\begin{gather*}
    \therefore \quad V_n = \frac{1}{1+r_n} \left[ \tilde p_n V_{n+1}(H)+ \tilde q_n V_{n+1}(T)\right]
\end{gather*}

This determines a backwards recursive relation for determining $V_0$, given $V_N$, the payoffs at time $N$.

\rightline{$\square$}

\vspace{5mm}
\noindent
(ii)$\;$
Subtracting gives (2) from (1) above gives $(u_n-d_n)\Delta_n S_n = X_{n+1}(H)- X_{n+1}(T)$, hence:
\begin{gather*}
    \Delta_n = \frac{X_{n+1}(H)- X_{n+1}(T)}{(u_n-d_n)S_n} 
\end{gather*}

\rightline{$\square$}

\vspace{5mm}
\noindent
(iii)$\;$ We first derive a formula for $v_n(s)$.
The possible prices at time five are: 130, 110, 90, 70, 50, and 30.
Since $v_5(s) = (s-80)^+$ we have that $v_5(s) = 0$ for $s \leq 80$.
\begin{gather*}
    v_5(130) = 50, \quad v_5(110) = 30, \quad v_5(90)=10, \quad
    v_5(70)=v_5(50)=v_5(30) =0
\end{gather*}
We have $u_n = \frac{S_{n+1}(H)}{S_n} = \frac{S_n+10}{S_n} = 1 + \frac{10}{S_n}$ and $d_n =\frac{S_{n+1}(T)}{S_n} = \frac{S_n-10}{S_n} = 1 - \frac{10}{S_n}$
Furthermore $r_n=0$ for all $n$.
We calculate:
\begin{gather*}
    \tilde p_n = \frac{ 1 + (0) - \left( 1 - \frac{10}{S_n}\right)}{\left( 1 + \frac{10}{S_n}\right) - \left( 1 - \frac{10}{S_n}\right)} = \frac{\left( \frac{10}{S_n}\right)}{\left( \frac{20}{S_n}\right)} = \frac{1}{2} \quad \implies \quad \tilde q_n = \frac{1}{2}
\end{gather*}
Therefore the recursive formula for $v_n$ is:
\begin{align*}
    v_n(s) &= \frac{1}{1+(0)}\left[\left( \tfrac{1}{2}\right) v_{n+1}(s+10) + \left(\tfrac{1}{2}\right) v_{n+1}(s-10) \right] \\
    &= \tfrac{1}{2}\left[v_{n+1}(s+10) + v_{n+1}(s-10) \right]
\end{align*}
Therefore:
\begin{align*}
    v_4(120) &= \tfrac{1}{2}\left[ (50) + (30) \right] = 40 &
    v_4(100) &= \tfrac{1}{2}\left[ (30) + (10) \right] = 20 \\
    v_4(80) &= \tfrac{1}{2}\left[ (10) + (0) \right] = 5 &
    v_4(60) &= v_4(40) = 0
\end{align*}
\begin{align*}
    v_3(110) &= \tfrac{1}{2}\left[ (40) + (20) \right] = 30 &
    v_3(90) &= \tfrac{1}{2}\left[ (20) + (5) \right] = \tfrac{25}{2} \\
    v_3(70) &= \tfrac{1}{2}\left[ (5) + (0) \right] = \tfrac{5}{2} &
    v_3(50) &= 0
\end{align*}
\begin{align*}
    v_2(100) &= \tfrac{1}{2}\left[ (30) + \left( \tfrac{25}{2} \right) \right] = \tfrac{85}{4} &
    v_2(80) &= \tfrac{1}{2}\left[ \left( \tfrac{25}{2} \right) + \left( \tfrac{5}{2} \right) \right] = \tfrac{15}{2} \\
    v_2(60) &= \tfrac{1}{2}\left[ \left( \tfrac{5}{2} \right) + \left( 0 \right) \right] = \tfrac{5}{4} 
\end{align*}
\begin{align*}
    v_1(90) = \tfrac{1}{2}\left[ \left( \tfrac{85}{4} \right) + \left( \tfrac{15}{2} \right) \right] = \tfrac{115}{8} \quad\quad\quad
    v_1(70) = \tfrac{1}{2}\left[ \left( \tfrac{15}{2} \right) + \left( \tfrac{5}{4} \right) \right] = \tfrac{35}{8}
\end{align*}
\begin{gather*}
    \therefore \quad v_0(80) = \tfrac{1}{2}\left[ \left( \tfrac{115}{8} \right) + \left( \tfrac{35}{8} \right) \right] = \tfrac{150}{16} = 9.375
\end{gather*}
The price at time zero of the option is $V_0 = 9.375$

\rightline{$\square$}

\end{document}
